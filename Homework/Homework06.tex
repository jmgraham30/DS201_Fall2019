\documentclass[12pt]{article}\usepackage[]{graphicx}\usepackage[]{color}
% maxwidth is the original width if it is less than linewidth
% otherwise use linewidth (to make sure the graphics do not exceed the margin)
\makeatletter
\def\maxwidth{ %
  \ifdim\Gin@nat@width>\linewidth
    \linewidth
  \else
    \Gin@nat@width
  \fi
}
\makeatother

\definecolor{fgcolor}{rgb}{0.345, 0.345, 0.345}
\newcommand{\hlnum}[1]{\textcolor[rgb]{0.686,0.059,0.569}{#1}}%
\newcommand{\hlstr}[1]{\textcolor[rgb]{0.192,0.494,0.8}{#1}}%
\newcommand{\hlcom}[1]{\textcolor[rgb]{0.678,0.584,0.686}{\textit{#1}}}%
\newcommand{\hlopt}[1]{\textcolor[rgb]{0,0,0}{#1}}%
\newcommand{\hlstd}[1]{\textcolor[rgb]{0.345,0.345,0.345}{#1}}%
\newcommand{\hlkwa}[1]{\textcolor[rgb]{0.161,0.373,0.58}{\textbf{#1}}}%
\newcommand{\hlkwb}[1]{\textcolor[rgb]{0.69,0.353,0.396}{#1}}%
\newcommand{\hlkwc}[1]{\textcolor[rgb]{0.333,0.667,0.333}{#1}}%
\newcommand{\hlkwd}[1]{\textcolor[rgb]{0.737,0.353,0.396}{\textbf{#1}}}%
\let\hlipl\hlkwb

\usepackage{framed}
\makeatletter
\newenvironment{kframe}{%
 \def\at@end@of@kframe{}%
 \ifinner\ifhmode%
  \def\at@end@of@kframe{\end{minipage}}%
  \begin{minipage}{\columnwidth}%
 \fi\fi%
 \def\FrameCommand##1{\hskip\@totalleftmargin \hskip-\fboxsep
 \colorbox{shadecolor}{##1}\hskip-\fboxsep
     % There is no \\@totalrightmargin, so:
     \hskip-\linewidth \hskip-\@totalleftmargin \hskip\columnwidth}%
 \MakeFramed {\advance\hsize-\width
   \@totalleftmargin\z@ \linewidth\hsize
   \@setminipage}}%
 {\par\unskip\endMakeFramed%
 \at@end@of@kframe}
\makeatother

\definecolor{shadecolor}{rgb}{.97, .97, .97}
\definecolor{messagecolor}{rgb}{0, 0, 0}
\definecolor{warningcolor}{rgb}{1, 0, 1}
\definecolor{errorcolor}{rgb}{1, 0, 0}
\newenvironment{knitrout}{}{} % an empty environment to be redefined in TeX

\usepackage{alltt}
\textwidth=7in
\textheight=9.5in
\topmargin=-1in
\headheight=0in
\headsep=.5in
\hoffset  -.85in

\pagestyle{empty}

\usepackage{amsmath,amssymb,amsfonts}
\usepackage{url}

\renewcommand{\thefootnote}{\fnsymbol{footnote}}
\IfFileExists{upquote.sty}{\usepackage{upquote}}{}
\begin{document}

\begin{center}
{\bf Intro to Data Science \\ Homework 6: Due Wednesday October 30 at 2:00pm}
\end{center}

\setlength{\unitlength}{1in}

\begin{picture}(6,.1)
\put(0,0) {\line(1,0){6.5}}
\end{picture}

\renewcommand{\arraystretch}{2}

\vskip.25in

\noindent{\bf  {\Large Exercises:} }



\vskip.25in

  \begin{enumerate}
    \item This exercise involves the use of simple linear regression on the Auto dataset from the ISLR package. 
    \begin{enumerate}
      \item  Create a scatter plot of the variables mpg versus horsepower from the Auto data set.
      \item Describe your observations from the scatter plot. 
      \item What does the following command do and how should you interpret the result? 
\begin{knitrout}
\definecolor{shadecolor}{rgb}{0.969, 0.969, 0.969}\color{fgcolor}\begin{kframe}
\begin{alltt}
\hlkwd{with}\hlstd{(Auto,}\hlkwd{cor}\hlstd{(horsepower,mpg))}
\end{alltt}
\begin{verbatim}
## [1] -0.7784268
\end{verbatim}
\end{kframe}
\end{knitrout}
      \item Use the lm() function to perform a simple linear regression with mpg as the response and horsepower as the predictor. That is, use the following command:
\begin{knitrout}
\definecolor{shadecolor}{rgb}{0.969, 0.969, 0.969}\color{fgcolor}\begin{kframe}
\begin{alltt}
\hlstd{auto_fit} \hlkwb{<-} \hlkwd{lm}\hlstd{(mpg}\hlopt{~}\hlstd{horsepower,}\hlkwc{data}\hlstd{=Auto)}
\end{alltt}
\end{kframe}
\end{knitrout}
      \item What do you learn from the information output by the following commands:
\begin{knitrout}
\definecolor{shadecolor}{rgb}{0.969, 0.969, 0.969}\color{fgcolor}\begin{kframe}
\begin{alltt}
 \hlkwd{tidy}\hlstd{(auto_fit)}
\end{alltt}
\begin{verbatim}
## # A tibble: 2 x 5
##   term        estimate std.error statistic   p.value
##   <chr>          <dbl>     <dbl>     <dbl>     <dbl>
## 1 (Intercept)   39.9     0.717        55.7 1.22e-187
## 2 horsepower    -0.158   0.00645     -24.5 7.03e- 81
\end{verbatim}
\end{kframe}
\end{knitrout}
      and
\begin{knitrout}
\definecolor{shadecolor}{rgb}{0.969, 0.969, 0.969}\color{fgcolor}\begin{kframe}
\begin{alltt}
 \hlkwd{glance}\hlstd{(auto_fit)}
\end{alltt}
\begin{verbatim}
## # A tibble: 1 x 11
##   r.squared adj.r.squared sigma statistic  p.value    df logLik   AIC   BIC
##       <dbl>         <dbl> <dbl>     <dbl>    <dbl> <int>  <dbl> <dbl> <dbl>
## 1     0.606         0.605  4.91      600. 7.03e-81     2 -1179. 2363. 2375.
## # ... with 2 more variables: deviance <dbl>, df.residual <int>
\end{verbatim}
\end{kframe}
\end{knitrout}
      
      \item Make a plot of the residuals versus the fitted values from the regression. Recall that you can use the augment function in the broom package to create a data frame that adds the residuals and fitted values from the linear regression to the original data set. What do you conclude from this plot?  
      \item Compute the MSE and RMSE from the regression. 
      \item Is there a relationship between the predictor and the response? 
      \item How strong is the relationship between the predictor and the response? 
      \item Is the relationship between the predictor and the response positive or negative? 
      \item Plot the linear regression (remember that you can use geom smooth for this) along with the scatterplot of the data. What do you observe from this plot?  
    \end{enumerate}
    
   \item Use a bootstrap to approximate a 95\% confidence interval for the slope parameter in the linear regression from problem 1. Be sure to make a plot of the bootstrap distribution. 
    
    \item Use a permutation test to test the null hypothesis: $H_{0}:$ slope is zero versus the alternative hypothesis $H_{A}:$ slope is not equal to zero in the regression fit for problem 1. A plot is probably very helpful here. 
    
   
  \end{enumerate}
    
\end{document}
