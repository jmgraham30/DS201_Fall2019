\documentclass[12pt]{article}
\textwidth=7in
\textheight=9.5in
\topmargin=-1in
\headheight=0in
\headsep=.5in
\hoffset  -.85in

\pagestyle{empty}

\usepackage{amsmath,amssymb,amsfonts}
\usepackage{url}
\usepackage{graphicx}

\renewcommand{\thefootnote}{\fnsymbol{footnote}}
\begin{document}

\begin{center}
{\bf Introduction to Data Science \\ Homework 2: Due Wednesday September 18 at 2:00pm}
\end{center}

\setlength{\unitlength}{1in}

\begin{picture}(6,.1)
\put(0,0) {\line(1,0){6.5}}
\end{picture}

\renewcommand{\arraystretch}{2}

\vskip.25in

\noindent{\bf  {\Large Exercises:} }

\vskip.25in
  \begin{enumerate}
    \item Visit \url{http://data.gov}, and identify five data sets that sound interesting to you. For each write a brief description
    and propose three interesting things you might do with them.
    \item Read the first few sections through \emph{Visualising distributions} of \emph{R for Data Science} \url{https://r4ds.had.co.nz/}.
    \item True or False: The {\tt diamonds} data set in the {\tt ggplot2} package is a data frame. One way to answer this question is by typing the command {\tt str(diamonds)} in R\footnote{If you want to look at the {\tt diamonds} data set, make sure you have the {\tt ggplot2} package installed and loaded.}.
    \item How many variables are there in the {\tt diamonds} data set?
    \item Classify each of the variables in the {\tt diamonds} data set. That is, state if the variable is continuous, discrete, categorical, etc.
    \item Describe the difference between a bar plot and a histogram. Under what circumstances would you use each?
    \item Explain the result of the command
    \begin{verbatim}
      ggplot(data = mpg) + geom_bar(mapping = aes(x=class))
    \end{verbatim}
    Make sure to answer the question in the context of the data. 
  \item Explain the result of the command
    \begin{verbatim}
      ggplot(data = mpg) + geom_histogram(mapping = aes(x=displ),binwidth=0.4)
    \end{verbatim}
     Make sure to answer the question in the context of the data. 
 
 \item Explain the meaning of the results obtained after running the R command\footnote{Make sure you have the {\tt ggplot2} package installed and loaded.}
     \begin{verbatim}
         summary(mpg)
     \end{verbatim}
     What information does this tell you about the {\tt mpg} data set?
     
     \item Describe what each of the listed functions do, be detailed
        \begin{itemize}
          \item {\tt  list.files}
          \item {\tt  dir.create}
          \item {\tt  file.create}
          \item {\tt  file.info}
          \item {\tt  file.rename}
          \item {\tt  file.copy}
        \end{itemize} 
    \item Read through section 4 of the article \emph{Excuse me, do you have a moment to talk
about version control?}, then answer the following questions. 
\begin{enumerate}
  \item What is Git?
  \item What is GitHub? 
  \item Why should someone in data science care about these? 
\end{enumerate} 
     
  \end{enumerate}

\end{document}
